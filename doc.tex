\documentclass[a4paper,twoside=false,abstract=false,numbers=noenddot,
titlepage=false,headings=small,parskip=half,version=last]{scrartcl}
\usepackage[utf8]{inputenc}
\usepackage[T1]{fontenc}
\usepackage[english]{babel}
\usepackage[colorlinks=true, pdfstartview=FitV,
linkcolor=black, citecolor=black, urlcolor=blue]{hyperref}
\usepackage{verbatim}
\usepackage{graphicx}
\usepackage{multirow}
\usepackage{color}

\usepackage{tikz}
\usetikzlibrary{matrix}

\usepackage{amsmath}
\usepackage{amsthm}
\usepackage{amssymb}
\usepackage{amsfonts}

\usepackage{gensymb}

\usepackage{authblk}

\usepackage{helpers}


\title{Solid State Physics - IM2601}
\subtitle{Laboration 1}
    \author[1]{Fredrik Forsberg}
    \author[1]{Jim Holmström}
    \author[1]{Samuel Zackrisson}
    \affil[1]{Engineering Physics, Royal Institute of Technology}
    \affil[1]{\{fforsber, jimho, samuelz\}@kth.se}


\begin{document}
\maketitle
\thispagestyle{empty}

\section{Introduction}
The standard method for characterization of any material's crystalline structure is X-ray diffraction, also known as XRD.
By studying how the X-rays reflects from the crystalline material, one can get an understanding of the material's properties and structure.

In this laboratory we are faced with two main tasks.
Firstly, to compute the wavelength ($\lambda$) of the X-rays, by measuring the diffraction behavior of NaCl and comparing it to our theoretically calculated values.
Secondly, for four samples, labled A, B, C and D, determine either the material in question or the orientation of the crystal plane of said material.

\section{Experiment procedure}
\subsection{$\boldsymbol{\beta}\boldsymbol{-}\boldsymbol{2}\boldsymbol{\beta}$ scan}
We can describe the reflection of an X-ray beam by treating it as specular (mirrorlike) and assuming elastic scattering. Based on these premises Bragg's law holds \cite{Kittel}.
\begin{equation}
    \label{eq:braggs}
    2 \cdot d(hkl) \cdot sin( \theta ) = \lambda
\end{equation}

$d(hkl)$ is the distance between parallel lattice planes of the configuration $(hkl)$
characterizing the reciprocal lattice vector $\bold{G}=\frac{2\pi}{a}(h,k,l)$, where $a$ is the lattice constant.
$d(hkl)$ can be normalized and shown to be equal to $\frac{2\pi}{\abs{\bold{G}}} = \frac{2\pi}{a}\sqrt{h^2+k^2+l^2}$ \cite{Kittel}.
In this laboratory we use a $\beta-2\beta$ scan, also known as coupled mode scan.
The incident beam hits the sample at an angle $\beta$, which is the same as the angle of the reflecting beam.
This is done by rotating the sample to an angle $\beta$ and subsequently moving the detector to an angle of $2\beta$, as shown in figure 1.

\image{uppstallning}{The $\beta-2\beta$ scan setup and visualization of corresponding wave vector in reciprocal space}
The scattering vector $\bold{q}$ equals to the difference between the outgoing and incoming wavevectors,
$\bold{k}' - \bold{k}$. By equation (20) in chapter 2 from Kittel \cite{Kittel} the scattering
amplitude $F$ is theoretically found to be
\begin{equation}
    \label{eq:scatamp}
    F = \sum_\bold{G} \int dV n_{\bold{G}} e^{i(\bold{G}-\bold{q})\cdot r}
\end{equation}

When the scattering vector equals a particular reciprocal lattice vector, $\bold{q} = \bold{G}$, the intensity of the reflection will peak in the diffractogram if the structure factor allows it.
This occurs since when $\bold{q}$ differs significally from $\bold{G}$, $F$ will becomes negligibly small.


In bounded mode the scattering vector $\bold{q}$ is altered only by the single angle $\beta$.
The length of the scattering vector can be determined from figure 1 as $\abs{\bold{q}} = 2 \abs{\bold{k}} sin(\beta)$.
This means in that $\bold{q}$ is bounded by $\beta$ which creates the limitation $2 \abs{\bold{k}} sin(\beta_{min}) < \abs{\bold{q}} < 2 \abs{\bold{k}} sin(\beta_{max})$.
A great limitation is furthermore that in coupled mode, only beams reflected off planes with normals parallel to the scattering vector might be measured by the detector. The covered space of the reciprocal space is therefore only a limited curve.
If we didn't have coupled mode on, then we could have freely moved the detector for every angle $\beta$ and therefore gained more information and another degree of freedom.


A Leybold Didactic diffractometer (no. 554 811) was used in this laboratory. The measurements were all made in coupled mode (i.e $\beta-2\beta$ scan). The default settings for the diffractometer were a voltage of 35 kV (maximum), a current of 1 mA (minimum) and a step angle of 0.1$\degree$ (minimal). For the NaCl measuremnet, a time step of 1 s and incident angles from 2.5$\degree$ to 30$\degree$ were used. The other samples were measured with a time step of 4 s and incident angles from 2.5$\degree$ to 35$\degree$. The measurement result of the reflected intensity and its corresponding incident angle $\beta$ was automatically recorded during the laboratory. The peaks were thereafter manually detected by searching for distinct local maxima.


\subsection{Structure factors}
The second condition for a diffraction peak to appear is that the structure factor must be nonzero.
The structure factor $S_{\bold{G}}$ of a crystal structure can be calculated as below. Here $n$ is the number of atoms in the basic lattice, and atom $j$ is at position $\bold{r}_j$ with atomic form factor $f_j$. The atomic form factors are approximately equal to the atomic number.\\
\begin{equation}
    S_{\bold{G}} = \sum^n_{j=1} f_j e^{-i \left( \bold{r}_j \cdot \bold{G}\right)}
\end{equation}

Say $\bold{G} = \frac{2\pi}{a}\left( h,k,l \right)$. The structure factor of an fcc lattice with lattice constant $a$ is
\begin{equation}
    S^{fcc}_{\bold{G}} = f\cdot \left[ 1 + (-1)^{h+k} + (-1)^{k+l} + (-1)^{l+h}\right]
\end{equation}
The fcc structure factor is nonzero if and only if $h$, $k$ and $l$ are all even or all odd.\\

\subsubsection{NaCl}
The two atoms in the conventional cell of NaCl, one Na and one Cl, are each in an fcc structure.
Putting an Na at the origin, the Cl structure will be displaced by $\frac{a}{2}\bold{e}_x$.
The structure factor sums can then be expressed using the fcc structure factor above as
\begin{align}
    S^{NaCl}_{\bold{G}}
    &= \left( f_{Na} + f_{Cl}e^{2\pi i\cdot \frac{h}{2}} \right) \cdot \left[ 1 + (-1)^{h+k} + (-1)^{k+l} + (-1)^{l+h}\right]\nonumber\\
    &= \left( f_{Na} + f_{Cl}(-1)^{h} \right) \cdot S^{fcc}_{\bold{G}}
\end{align}\label{naclstructure}
With $f_{Na}=11$ and $f_{Cl}=17$ the structure factor will only be $0$ when the fcc part is $0$.

\subsubsection{Si}
Silicon is monoatomic with two atoms in the conventional cell, at $(0,0,0)$ and at $a(\frac{1}{4},\frac{1}{4},\frac{1}{4})$.
The structure factor becomes

\begin{equation}
    \label{eq:sistructure}
    S^{Si}_{\bold{G}}
    = f_{Si}\cdot \left( 1 + e^{\pi i \frac{h+k+l}{2}} \right) \cdot S^{fcc}_{\bold{G}}
\end{equation}

\textbf{GaAs and InP}\\
The other two crystals in this lab both have a zincblende structure with two different atomic factors , say $f_\alpha,f_\beta$.
The $\alpha$-atoms are in an fcc structure starting at the origin and the $\beta$-atoms are displaced by $\frac{1}{4}\frac{1}{4}\frac{1}{4}$, again in a cubic fcc structure.
The structure factor becomes

\begin{equation}
    \label{eq:zincblendestructure}
    S^{\alpha\beta}_{\bold{G}}
    = \left( f_\alpha + f_\beta e^{\pi i \frac{h+k+l}{2}} \right) \cdot S^{fcc}_{\bold{G}}
\end{equation}

The atomic form factors are $f_{Ga}=31$, $f_{As}=33$, $f_{In}=49$ and $f_{P}=15$.
The structure factor for InP is only zero if and only if the fcc structure factor is zero.
The same goes for GaAs, except when $h+k+l\equiv_4 2$ the structure factor becomes small which might affect the diffraction peaks.

\section{Measurements results}
\plot{A}{Description A}
\plot{B}{Description B}
\plot{C}{Description C}
\plot{D}{Description D}
\plot{NaCl}{Description NaCl}

\begin{tabular}{ |l|l| }
    \hline
    \multicolumn{2}{|c|}{Sample A} \\
    \hline
    $\beta_1$ & 6.5\degree \\
    $\beta_2$ & 19.8\degree \\
    $\beta_3$ & 26.9\degree \\
    ($\beta_4$ & 34.4\degree) \\
    \hline
\end{tabular}

\begin{tabular}{ |l|l| }
    \hline
    \multicolumn{2}{|c|}{Sample B} \\
    \hline
    $\beta_1$ & 14.7\degree \\
    ($\beta_2$ & 30.2\degree) \\
    \hline
\end{tabular}

\begin{tabular}{ |l|l| }
    \hline
    \multicolumn{2}{|c|}{Sample C} \\
    \hline
    $\beta_1$ & 15.1\degree \\
    $\beta_2$ & 27.6\degree \\
    $\beta_3$ & 31.4\degree \\
    \hline
\end{tabular}

\begin{tabular}{ |l|l| }
    \hline
    \multicolumn{2}{|c|}{Sample D} \\
    \hline
    $\beta_1$ & 6.0\degree \\
    $\beta_2$ & 12.1\degree \\
    $\beta_3$ & 18.3\degree \\
    $\beta_4$ & 24.8\degree \\
    \hline
\end{tabular}

\begin{tabular}{ |l|l| }
    \hline
    \multicolumn{2}{|c|}{Sample NaCl} \\
    \hline
    $\beta_1$ & 7.2\degree \\
    $\beta_2$ & 14.5\degree \\
    $\beta_3$ & 22.1\degree \\
    \hline
\end{tabular}

\section{Discussion of the results}
\subsection{Finding the X-ray wavelength from the NaCl diffraction pattern}
The crystal surface is cut parallel to the $(1,0,0)$ lattice plane.
In coupled mode the $hkl$-vectors will be integer multiples of this Miller index.
$\bold{G}=\frac{2\pi}{a}(m,0,0)$ where $m$ is an integer.
The structure factor for NaCl is nonzero when $h,k,l$ are all even or all odd.
This happens when $m$ is even.
From \eqref{braggs} our diffraction maxima occurs when
\begin{equation}
\frac{2a}{n}sin\theta=\lambda\nonumber
\end{equation}
where $a=5.63\text{ Å}$.\\
As the (the X-ray wavelength) is constant an increasing angle comes with increasing $m$.
The first maxmimum occurs for $m=2$, the second for $m=4$ et cetera.
Inputting the measured peak angles gives the following values for $\lambda$:\\

\begin{tabular}{ |l|l|l|l| }
    \hline
    \multicolumn{2}{|c|}{Angle}& $m$ & $\lambda$\\
    \hline
    $\beta_1$	& 7.2\degree	& 2	& 0.7056 Å	\\
    $\beta_2$	& 14.5\degree	& 4	& 0.7048	\\
    $\beta_3$	& 22.1\degree	& 6	& 0.7060	\\
	\hline
	\multicolumn{3}{r|}{mean:}	& 0.7055	\\
    \cline{4-4}
\end{tabular}

Given the X-ray wavelength, equation \eqref{braggs} can be used to find the expected angles for different materials and crystal cuts.

\subsection{Determining the materials of sample A and D}

As the crystal samples A and D are cut with Miller index $(1,1,1)$,
the integer triple $(h,k,l)=m\cdot (1,1,1)$ for integers $m$.

With $\lambda$ in equation \eqref{eq:braggs} known, the equation can be rearranged to find expected angles for the a crystal.
\begin{align}
\label{eq:braggangle}
\theta&=sin^{-1}\left( \frac{\lambda}{2a} \sqrt{h^2+k^2+l^2}\right)\\
&=sin^{-1}\left( \frac{\sqrt{3}\lambda}{2a} \cdot m\right)\nonumber
\end{align}
Which of these angles give peaks depend on the structure factor of the sample.
For InP the structure factor is always nonzero along this plane - $h,k,l$ always have the same parity.
The structure factor of Si disappears when $h+k+l\equiv_4 2$, which happens for $m=2,6,10,...$.
GaAs is expected to have no or small peaks for this case, as the structure factor becomes very small due to the atomic form factors.
The expected angles are compared to the experimental measurements in the table below.

\begin{tabular}{ |c|c|c|c|c|c|c|c|c|c| }
    \hline
    \multicolumn{4}{|c|}{$\theta_{obs}$}
	& \multicolumn{2}{|c|}{InP}
	& \multicolumn{2}{|c|}{Si}
	& \multicolumn{2}{|c|}{GaAs}\\
    \hline
	\multicolumn{2}{|c|}{Sample A} & \multicolumn{2}{|c|}{Sample B} & $m$ & $\theta$ & $m$ & $\theta$ & $m$ & $\theta$ \\
	\hline
    $\beta_1$	& 6.5\degree	& $\beta_1$	& 6.0\degree	& 1 & 5.975\degree & 1	& 6.461\degree &  1		& 6.208\degree	\\
    $\beta_2$	& 19.8\degree	& $\beta_2$	& 12.1\degree	& 2 & 12.02\degree & 3	& 19.73\degree & (2)	& 12.49\degree	\\
    $\beta_3$	& 26.9\degree	& $\beta_3$	& 18.3\degree	& 3 & 18.20\degree & 4	& 26.74\degree &  3		& 18.93\degree	\\
    $\beta_4$	& 34.4\degree	& $\beta_4$	& 24.8\degree	& 4 & 24.60\degree & 5	& 34.24\degree &  4		& 25.63\degree	\\
    -			& -				& -			& -				& 5 & 31.36\degree & 7	& 51.97\degree &  5		& 32.73\degree	\\
	\hline
\end{tabular}

The $m=1,3,4,5$ peaks of Si fit very neatly with the experimental data for sample A, while the InP peaks fit poorly.
Nothing is visible in the sample A data near InP:s $m=2$-peak.
Sample A is cleary Si.\\
Sample B fits with InP and GaAs, but better with InP. The peak at $\beta_2$ would be diminished if it was a GaAs sample, as the structure factor would be very low. 

\subsection{Determining the crystal planes of samples B and C}

Given the material and the wavelength, one can investigate along which plane a crystal was cut.
These diamond/zincblende crystals must be cut along planes with Miller indices $(1,0,0)$, $(1,1,0)$ or $(1,1,1)$, of which the $(h,k,l)$-vectors then are multiples.
For example, the $(1,1,0)$-plane gives $\sqrt{h^2+k^2+l^2}=\sqrt{2},\sqrt{4},\sqrt{6}$ et cetera.

The first angles for which the structure factor is nonzero are plotted below for each crystal.
Theoretical GaAs peaks for which the structure factor is almost 0 are within parantheses.

\begin{tabular}{ |c|c|c|c|c|c|c|c| }
	\hline
    \multicolumn{8}{|c|}{GaAs}\\
    \hline
    \multicolumn{2}{|c|}{$\theta_{obs}$}
	& \multicolumn{2}{|c|}{$(m,0,0)$}
	& \multicolumn{2}{|c|}{$(m,m,0)$}
	& \multicolumn{2}{|c|}{$(m,m,m)$}\\
    \hline
	\multicolumn{2}{|c|}{Sample B}& $m$ & $\theta$ & $m$ & $\theta$ & $m$ & $\theta$ \\
	\hline
    $\beta_1$	& 14.7\degree	& (2)	& 7.173\degree & 2	& 10.17\degree & 1	& 6.208\degree	\\
    $\beta_2$	& 30.2\degree	& 4		& 14.46\degree & 4	& 20.68\degree & (2)	& 12.49\degree	\\
    -			& -				& (6)	& 22.00\degree & 6	& 32.00\degree & 3	& 18.93\degree	\\
    -			& -				& 8		& 29.96\degree & 8	& 44.94\degree & 4	& 25.63\degree	\\
	\hline
\end{tabular}


\begin{tabular}{ |c|c|c|c|c|c|c|c| }
	\hline
    \multicolumn{8}{|c|}{Si}\\
    \hline
    \multicolumn{2}{|c|}{$\theta_{obs}$}
	& \multicolumn{2}{|c|}{$(m,0,0)$}
	& \multicolumn{2}{|c|}{$(m,m,0)$}
	& \multicolumn{2}{|c|}{$(m,m,m)$}\\
    \hline
	\multicolumn{2}{|c|}{Sample C}& $m$ & $\theta$ & $m$ & $\theta$ & $m$ & $\theta$ \\
	\hline
    $\beta_1$	& 15.1\degree	& 4		& 15.06\degree	& 2	& 10.59\degree	& 1	& 6.461\degree	\\
    $\beta_2$	& 27.6\degree	& 8		& 31.31\degree	& 4	& 21.56\degree	& 3	& 19.73\degree	\\
    $\beta_3$	& 31.4\degree	& 12	& 51.22\degree	& 6	& 33.45\degree	& 4	& 26.75\degree	\\
    -			& -				& 16	& -				& 8	& 47.31\degree	& 5	& 34.24\degree	\\
	\hline
\end{tabular}

The obvious best fit for GaAs is the $(1,0,0)$-plane.
The possible low-intensity peaks for $m=2$, $m=4$ are, if they're even there, lost in the measurement noise.\\
Sample D fits best with $(1,0,0)$ as well.
The unexplained, small peak $\beta_2$ corresponds well with $m=6$,
where the theoretical structure factor is zero.
$m=6$ is at least diminished,
as the peaks do not follow the monotone-decreasing pattern of other plots.

\section{Conclusion}
It is a fairly straightforward procedure to determine which sample is which given the very limiting restraints on the problem.
Had the crystal materials not been limited it would have been a much harder problem. Even as it is, it is not trivial.
The mystery peak $\beta_2$ could be due to the mentioned diminished (theoretically zero) structure factor, or just from random noise.\\
$\beta-2\beta$ is a simple yet powerful method for examining the structure of crystal samples.

\begin{thebibliography}{1}
    \bibitem{Kittel}
        Charles Kittel,
        {\em Introduction to Solid State Physics 8th Edition},
        2005.
\end{thebibliography}


\definecolor{braggred}{rgb}{1.00, 1.00, 1.00}
\textcolor{braggred}{By Bragg's beard!}

\end{document}
