\documentclass[a4paper,twoside=false,abstract=false,numbers=noenddot,
titlepage=false,headings=small,parskip=half,version=last]{scrartcl}
\usepackage[utf8]{inputenc}
\usepackage[T1]{fontenc}
\usepackage[english]{babel}
\usepackage[colorlinks=true, pdfstartview=FitV,
linkcolor=black, citecolor=black, urlcolor=blue]{hyperref}
\usepackage{verbatim}
\usepackage{graphicx}
\usepackage{multirow}

\usepackage{tikz}
\usetikzlibrary{matrix}

\usepackage{amsmath}
\usepackage{amsthm}
\usepackage{amssymb}
\usepackage{amsfonts}

\usepackage{authblk}

\usepackage{helpers}


\title{Solid State Physics - IM2601}
\subtitle{Laboration 1}
    \author[1]{Fredrik Forsberg}
    \author[1]{Jim Holmström}
    \author[1]{Samuel Zackrisson}
    \affil[1]{Engineering Physics, Royal Institute of Technology}
    \affil[1]{\{fforsber, jimho, samuelz\}@kth.se}


\begin{document}
\maketitle
\thispagestyle{empty}

\section{Introduction}
The standard method for caracterization of any material's crystaline structure is X-ray diffraction, also known as XRD. By studying how the X-rays reflects from the crystaline material, one can get an understanding of the material's properties and structure.

In this laboratory we are faced with two main tasks. Firstly, to compute the wavelength ($\lambda$) of the X-rays, by measuring the diffraction behavior of NaCl and comparing it to our theoretically calculated values. Secondly, for four samples, labled A, B, C and D, determine either the material in question or the orientation of the crystal plane of said material.

\section{Experiment}
\subsection{$\beta-2\beta$ scan}
We can describe the reflection of an X-ray beam by treating it as specular (mirrorlike) and assuming elastic scattering. From these premises we can find Bragg's law:
\begin{equation}
    \label{eq:braggs}
    2 \cdot d(hkl) \cdot sin( \theta ) = \lambda
\end{equation}
$d(hkl)$ is the distance between parallel lattice planes of the configuration $(hkl)$. In this laboratory we use a $\beta-2\beta$ scan, also known as coupled mode. The incident beam hits the sample at an angle $\beta$, which is the same as the angle of the reflecting beam. This is done by rotating the sample to an angle $\beta$ and subsecuently moving the detector to an angle of $2\beta$, as shown in figure 1.

\image{uppstallning}{The $\beta-2\beta$ scan setup and visualization of corresponding wave vector in reciprocal space}




\plot{A}{Description A}
\plot{B}{Description B}
\plot{C}{Description C}
\plot{D}{Description D}
\plot{NaCl}{Description NaCl}


\section{Discussion}

\section{Conclusion}

\end{document}
